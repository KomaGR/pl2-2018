\documentclass[12pt]{article}
\usepackage{amsmath}
\usepackage{amssymb}
\usepackage[greek, english]{babel}
\usepackage{lmodern}
\usepackage{hyperref}

\allowdisplaybreaks


\newcommand{\whilelang}{\textrm{\textbf{WHILE}$^{cons}$\ }}
\newcommand{\gr}{\textgreek{}}


\title{Exercise 7: Denotational Semantics}
\author{Orestis Kaparounakis}
\date{\today}

\begin{document}
\maketitle
\begin{abstract}
    This is a report for Exercise 7, year 2018, of the course Programming
    Languages II (National Technical University of Athens).
    The exercise is available at \url{https://courses.softlab.ntua.gr/pl2/2018b/exercises/densem.pdf}.
    In this report the denotational semantics of the language \whilelang
    are defined.
\end{abstract}
\section{\textgreek{Γραμματική της} \whilelang}
\textgreek{Η αφηρημένη σύνταξη της} \whilelang
\textgreek{έχει ήδη οριστεί στην εκφώνηση της άσκησης αλλά
επαναλαμβάνεται εδώ για πληρότητα}. 

\vspace{-1em}
\begin{align*}
  C &::= \textrm{skip} \,|\, x ::= E \,|\, C_0\,;\,C_1
  \,|\, \textrm{if}\, E\, \textrm{then}\, C_0\, \textrm{else}\, C_1 
  \,|\, \textrm{for}\, E \, \textrm{do} \, C
  \,|\, \textrm{while}\, E\, \textrm{do}\, C. \\
  E &::= \textrm{0}  \,|\, \textrm{succ}\, E  \,|\, \textrm{pred} E
  \,|\, \textrm{true}  \,|\, \textrm{false}  \,|\, E_0 < E_1
  \,|\, E_0 = E_1  \,|\, \textrm{not}\, E \\
  &~\,|~~~ \textrm{if}\, E\, \textrm{then}\, C_0\, \textrm{else}\, C_1
  \,|\,E_0 : E_1  \,|\, \textrm{hd}\, E  \,|\, \textrm{tl}\, E.
\end{align*}

% \textgreek{Εφόσον πλέον δεν διαχωρίζουμε τις λογικές από τις 
% αριθημτικές εκφράσεις κάνουμε την ιστορικά διάσημη παραδοχή ότι
% σε} context \textgreek{λογικής έκφρασης θα είναι}
% $0=\textrm{false}$ \textgreek{ενώ οτιδήποτε άλλο θα αποτιμάται σε} true.

\section{\textgreek{Δηλωτική σημασιολογία της} \whilelang}
\subsection{\textgreek{Πεδίο των σημασιολογικών τιμών}}
\textgreek{Το πεδίο} (domain) \textgreek{των σημασιολογικών τιμών
των εκφράσεων, το οποίο θα συμβολίζουμε με $\mathbb{E}$, 
με μια πρώτη ματιά, ορίζεται αναδρομικά.
Αποτελείται από τα άτομα των πεδίων $\mathbb{N}$ και $\mathbb{T}$
\footnote{Τα πεδία αυτά ορίζονται στις διαφάνειες.}, 
αλλά και από ζεύγη} (cons cells)
\textgreek{τιμών, οι οποίες μπορεί να είναι είτε άτομα 
είτε άλλα ζεύγη. Δηλαδή το πεδίο $\mathbb{E}$ θα συγκροτείται
και από αντίγραφα του εαυτού του, είναι δηλαδή ανακλαστικό
} (reflexive).
\textgreek{Παρατηρούμε ότι ο ορισμός αυτός παίρνει την
μορφή δέντρου. Η πεδιακή εξίσωση (ισομορφισμός) που θέλουμε να μελετήσουμε εδώ,
σύμφωνα και με την ανάλυση του} D. Scott~\cite{scott1982domains},
\textgreek{είναι η}

\begin{equation}
  \label{eq:tree_rec}
  \mathbb{E} \cong \mathbb{A} + (\mathbb{E} \times \mathbb{E}).
\end{equation}

\textgreek{Αν υπάρχει τέτοιο πεδίο $\mathbb{E}$ τότε λέμε ότι τα στοιχεία 
του είναι είτε} bottom~($\bot$)
\textgreek{, είτε στοιχεία του $\mathbb{A}$, που αναπαριστά 
τα άτομα στοιχεία του πεδίου 
(με $\mathbb{A} = \mathbb{N} \cup \mathbb{T}$), 
είτε ζεύγη από στοιχεία του $\mathbb{E}$.

Αρχικά, θέλουμε να ορίσουμε τα άτομα του πεδίου $\mathbb{E}$.
Δηλαδή θέλουμε να βάλουμε ``αντίγραφα'' των στοιχείων του
$\mathbb{A}$. Ορίζουμε $\bot_E = (\bot, \bot)$, το στοιχείο του 
$\mathbb{E}$ που σε οποιαδήποτε περίπτωση πρέπει να υπάρχει.
Επομένως, ένα στοιχείο $X \in \mathbb{A}$ θα είναι
$(X,\bot)$ στο πεδίο $\mathbb{E}$.
Τα υπόλοιπα στοιχεία θα είναι της μορφής $(\bot, Y)$, όπου το
$Y \in \mathbb{U} ~|~ \mathbb{U} \cong \mathbb{E} \times \mathbb{E}$ 
αφορά τα άλλου είδους στοιχεία του $\mathbb{E}$. 
Έτσι, το πεδίο $\mathbb{E}$ ορίζεται σαν διαχωρισμένο 
άθροισμα των $\mathbb{A}$ και $\mathbb{U}$.
Το $\mathbb{U}$ ορίζεται ως το γινόμενο όλων των στοιχείων
που ανήκουν στο $\mathbb{E}$. 
θα είναι, λοιπόν, $(\bot, Y) \in \mathbb{E}$ για κάθε 
$Y \in \mathbb{U}$.
Έτσι, έχουμε τον επαγωγικό ορισμό του πεδίου $\mathbb{E}$:

\begin{enumerate}
  \item $\bot_E \in \mathbb{E}$,
  \item $(X,\bot) \in \mathbb{E}$ αν $X \in \mathbb{A}$,
  \item $(\bot, (Y,\bot_E)) \in \mathbb{E}$ και 
  $(\bot, (\bot_E, Z)) \in \mathbb{E}$ 
  για κάθε $Y,Z \in \mathbb{E}$.
\end{enumerate}}

\textgreek{
Μένει να ορίσουμε την διάταξη $\sqsubseteq_E$ 
για το πεδίο $\mathbb{E}$. Από την συνέχεια της ανάλυσης 
στο~\cite{scott1982domains} έχουμε:

\begin{enumerate}
  \item $\bot \sqsubseteq_E u$ για κάθε $u$,
  \item $Y \sqsubseteq_E u \cup \{\bot_E\}$ 
  αν $Y \sqsubseteq_E u$,
  \item $(W, \bot) \sqsubseteq_E 
  \{(X,\bot)~|~X \in w\}$ αν $W \sqsubseteq_A w$,
  \item $(\bot,(X, \bot_E)) \sqsubseteq_E 
  \{(\bot, (Y, \bot_E)) ~|~ Y \in u\} \cup
  \{(\bot, (\bot_E, Z))\}$ αν $X \sqsubseteq_E u$,
  \item $((X, \bot_E),\bot) \sqsubseteq_E 
  \{(\bot, (Y, \bot_E)) ~|~ Y \in u\} \cup
  \{(\bot, (\bot_E, Z))\}$ αν $X \sqsubseteq_E u$.
\end{enumerate}
}

\subsection{\textgreek{Σημασιολογικές εξισώσεις}}
\textgreek{Ορίζουμε ξανά τις σημασιολογικές συναρτήσεις
από τις διαφάνειες:}

\vspace{-1em}
\begin{align*}
  \mathcal{C}[\![\mathbf{C}]\!] &: S \rightarrow S_\bot\\
  \mathcal{E}[\![\mathbf{E}]\!] &: S \rightarrow \mathbb{E}\\
\end{align*}

\vspace{-1em}
\textgreek{όπου πλεον οι σημασιολογικές τιμές ανήκουν 
στο πεδίο $\mathbb{E}$ και δεν διαχωρίζονται
με βάση τον τύπο τους.

Ορίζουμε τώρα τις σημασιολογικές εξισώσεις της
γλώσσας} \whilelang.

\begin{align*}
  \mathcal{C}[\![\textrm{skip}]\!]\,s &= s\\
\
  \mathcal{C}[\![\mathbf{x} := \mathbf{E}]\!]\,s &= 
    s[\mathbf{x} := \mathcal{E}[\![\mathbf{E}]\!]\,s]\\
\
  \mathcal{C}[\![\mathbf{C}_0;\mathbf{C}_1]\!]\,s &= 
    \mathcal{C}[\![\mathbf{C}_1]\!]^+
    (\mathcal{C}[\![\mathbf{C}_0]\!]\,s)\\
\
  \mathcal{C}[\![\textrm{if \textbf{E} 
                         then \textbf{C}$_0$
                         else \textbf{C}$_1$}]\!]\,s &=
    \begin{cases}
      \mathcal{C}[\![\mathbf{C}_0]\!]\,s, & 
      \textgreek{\textrm{εαν}}~
      \mathcal{E}[\![\mathbf{E}]\!]\,s = \mathit{true}\\
      \mathcal{C}[\![\mathbf{C}_1]\!]\,s, & 
      \textgreek{\textrm{διαφορετικά}}
    \end{cases}\\
\
  \mathcal{C}[\![\textrm{for \textbf{E} 
                         do \textbf{C}}]\!]\,s &=
    (\mathcal{C}[\![\mathbf{C}]\!]^+)^n(s),~
    \textgreek{\textrm{όπου}}~n = \mathcal{E}[\![\mathbf{E}]\!]\,s\\
\
  \mathcal{C}[\![\textrm{while \textbf{E} 
                         do \textbf{C}}]\!]\,s &=
  \textrm{fix}\,F\,s,~\textgreek{\textrm{όπου}}\\
  F f s &= \begin{cases}
    f(\mathcal{C}[\![\mathbf{C}]\!]\,s), & \textgreek{\textrm{εαν}}~
    \mathcal{E}[\![\mathbf{E}]\!]\,s = \mathit{true}\\
    s,  & \textgreek{\textrm{εαν}}~
    \mathcal{E}[\![\mathbf{E}]\!]\,s = \mathit{false}
    \end{cases}
\end{align*}

\begin{align*}
  \mathcal{E}[\![\textrm{0}]\!]\,s &= 0\\
\
  \mathcal{E}[\![\textrm{succ \textbf{E}}]\!]\,s &= 
    \mathcal{E}[\![\textrm{\textbf{E}}]\!]\,s + 1\\
\
  \mathcal{E}[\![\textrm{pred \textbf{E}}]\!]\,s &= 
    \mathcal{E}[\![\textrm{\textbf{E}}]\!]\,s - 1\\
\
  \mathcal{E}[\![\textrm{true}]\!]\,s &= \mathit{true}\\
\
  \mathcal{E}[\![\textrm{false}]\!]\,s &= \mathit{false}\\
\
  \mathcal{E}[\![\mathbf{E}_0<\mathbf{E}_1]\!]\,s &=
    \begin{cases}
      \mathit{true},  & \textgreek{\textrm{εαν}}~
      \mathcal{E}[\![\mathbf{E}_0]\!]\,s < \mathcal{E}[\![\mathbf{E}_1]\!]\,s\\      
      \mathit{false}, & \textgreek{\textrm{διαφορετικά}}      
    \end{cases}\\
\
  \mathcal{E}[\![\mathbf{E}_0=\mathbf{E}_1]\!]\,s &=
    \begin{cases}
      \mathit{true},  & \textgreek{\textrm{εαν}}~
      \mathcal{E}[\![\mathbf{E}_0]\!]\,s = \mathcal{E}[\![\mathbf{E}_1]\!]\,s\\      
      \mathit{false}, & \textgreek{\textrm{διαφορετικά}}      
    \end{cases}\\
\
  \mathcal{E}[\![\textrm{not}~\mathbf{E}]\!]\,s &=
    \begin{cases}
      \mathit{true},  & \textgreek{\textrm{εαν}}~
      \mathcal{E}[\![\mathbf{E}]\!]\,s = \mathit{false}\\     
      \mathit{false},  & \textgreek{\textrm{εαν}}~
      \mathcal{E}[\![\mathbf{E}]\!]\,s = \mathit{true}      
    \end{cases}\\
\
  \mathcal{E}[\![\textrm{if \textbf{E}$_0$
                        then \textbf{E}$_1$
                        else \textbf{E}$_2$}]\!]\,s &=
    \begin{cases}
      \mathcal{E}[\![\mathbf{E}_1]\!]\,s, & 
      \textgreek{\textrm{εαν}}~
      \mathcal{E}[\![\mathbf{E}_0]\!]\,s = \mathit{true}\\
      \mathcal{E}[\![\mathbf{E}_2]\!]\,s, & 
      \textgreek{\textrm{διαφορετικά}}
    \end{cases}\\
\
  \mathcal{E}[\![\mathbf{E}_0 : \mathbf{E}_1]\!]\,s &=
    (\bot, (\mathcal{E}[\![\mathbf{E}_0]\!]\,s,
    \mathcal{E}[\![\mathbf{E}_1]\!]\,s))\\
\
  \mathcal{E}[\![\textrm{hd}~\mathbf{E}]\!]\,s &=
  \begin{cases}
    \mathcal{E}[\![\mathbf{E}_0]\!]\,s, & 
    \textgreek{\textrm{εαν}}~
    % \mathcal{E}[\![\mathbf{E}]\!]\,s =
    % \mathcal{E}[\![\mathbf{E}_0 : \mathbf{E}_1]\!]\,s\\
    (\bot, (\mathcal{E}[\![\mathbf{E}_0]\!]\,s,
    \bot_E) \sqsubseteq_E
    \mathcal{E}[\![\mathbf{E}]\!]\,s\\
    \textgreek{\textrm{μη ορισμένο}}, & \textgreek{\textrm{διαφορετικά}}
  \end{cases}\\
\
  \mathcal{E}[\![\textrm{tl}~\mathbf{E}]\!]\,s &=
  \begin{cases}
    \mathcal{E}[\![\mathbf{E}_1]\!]\,s, & 
    \textgreek{\textrm{εαν}}~
    % \mathcal{E}[\![\mathbf{E}]\!]\,s =
    % \mathcal{E}[\![\mathbf{E}_0 : \mathbf{E}_1]\!]\,s\\
    (\bot, (\bot_E,
    \mathcal{E}[\![\mathbf{E}_1]\!]\,s) \sqsubseteq_E
    \mathcal{E}[\![\mathbf{E}]\!]\,s\\
    \textgreek{\textrm{μη ορισμένο}}, & \textgreek{\textrm{διαφορετικά}}
  \end{cases}\\
\end{align*}

\textgreek{όπου έχουν γίνει οι αντικαταστάσεις
$X \leftarrow (X, \bot)~|~X \in \mathbb{A}$,
δηλαδή όπου εμφανίζονται τα άτομα του πεδίου
$\mathbb{E}$, για συντομία και ευκολότερη ανάγνωση.}

\textgreek{Φυσικά, για να βγάζουν τα παραπάνω νόημα στο} 
runtime \textgreek{θα πρέπει να διασφαλίζεται δυναμικά
η ορθή χρήση των τύπων. Δηλαδή ότι δεν γίνεται
χρήση των λογικών τιμών εκεί που έχουν νόημα μόνο
αριθμητικές ή και το ανάποδο.}


\newpage
\nocite{*}
\bibliography{def}
\bibliographystyle{plain}
\end{document}