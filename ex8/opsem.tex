\documentclass[12pt]{article}
\usepackage{amsmath, amssymb}
\usepackage{semantic}
\usepackage[greek, english]{babel}
\usepackage{lmodern}
\usepackage{hyperref}

\newcommand{\spacing}{1em}
\allowdisplaybreaks

\title{Exercise 8: Type Systems}
\author{Orestis Kaparounakis\\
\small\textgreek{Α.Μ.}: 03114057, email: orestis.kapar@hotmail.com}
\date{\today}

\begin{document}
\maketitle
\begin{abstract}
    This is a report for Exercise 8, year 2018, of the course Programming
    Languages II (National Technical University of Athens).
    The exercise is available at \url{https://courses.softlab.ntua.gr/pl2/2018b/exercises/typesys.pdf}.
    In this report the big-step (or natural) semantics of the subject language
    are defined.
\end{abstract}

\setcounter{section}{-1}
\section{\textgreek{Σύνταξη της υποκείμενης γλώσσας}}
\textgreek{Παρακάτω συλλέγουμε την σύνταξη που προκύπτει για 
την υποκείμενη γλώσσα, λαμβάνοντας υπόψη τις διαφάνειες
4--12, 13--15 και 32--39.}

\vspace{-1em}
\begin{align*}
  \tau &::= \textrm{Int} \,|\, \textrm{Bool} \,|\, \tau_1 \rightarrow \tau_2 
  \,|\, \textrm{Ref}\, \tau.\\
\
  e &::= n  \,|\, -e  \,|\, e_1 + e_2
  \,|\, true  \,|\, false \,|\, \neg e \,|\, e_1 \wedge e_2 \,|\, e_0 < e_1
  \,|\,  \textrm{if}\, e\, \textrm{then}\, e_1\, \textrm{else}\, e_2\\
  &~~~|~~ x \,|\, \lambda x : \tau . e \,|\, e_1\,e_2
  \,|\, \textrm{ref}\, e \,|\, !e \,|\, e_1 := e_2 \,|\, \textrm{loc}_i.\\
\
  v &::= n \,|\, true \,|\, false \,|\, \lambda x : \tau . e 
    \,|\, \textrm{loc}_i.\\
\
  s &::= (e, m).
\end{align*}

\textgreek{Όπου, ο κανόνας των $\tau$ αναφέρεται στους τύπους που δύναται
αποκτήσουν οι εκφράσεις $e$, ενώ $v$ είναι αντίστοιχα οι τιμές. 
Το $s$ αναπαριστά το περιβάλλον αποτίμησης, όπου πλέον 
λαμβάνεται υπόψη η κατάστασης της μνήμης (ως $m$).}

\section{\textgreek{Σημασιολογία μεγάλων βημάτων}}
\label{sec:big-step-sos}
\textgreek{Αναπτύσουμε τώρα την λειτουργική σημασιολογία μεγάλων βημάτων
για την υποκείμενη γλώσσα. Ξεκινάμε με τα αξιώματα για τις τιμές $v$ και 
έπειτα προχωράμε στους συμπερασματικούς κανόνες για τις εκφράσεις $e$.
Στους κανόνες εμπλέκεται και η κατάσταση της μνήμης $m$, καθώς σε αρκετούς
μεταβάλλεται. Αυτό συμβαίνει επειδή κατά την αποτίμηση οποιασδήποτε
έκφρασης $e$ ενδέχεται να παρεμβληθούν δεσμεύσεις και συνωνυμίες.}

\begin{align*}
\inference[\textbf{Vnum:}]{}{
    \langle n, m \rangle \Downarrow \langle n \rangle
}
\end{align*}
\begin{align*}
&\inference[\textbf{Vtrue:}]{}{
    \langle true, m \rangle \Downarrow \langle true \rangle
}
&\inference[\textbf{Vfalse:}]{}{
    \langle false, m \rangle \Downarrow \langle false \rangle
}
\\[\spacing]
&\inference[\textbf{Vlambda:}]{}{
    \langle \lambda x : \tau . e, m \rangle \Downarrow \langle \lambda x : \tau . e \rangle
}
&\inference[\textbf{Vloc:}]{}{
    \langle \textrm{loc}_i, m \rangle \Downarrow \langle \textrm{loc}_i \rangle
}
\end{align*}
\vspace{2em}
\begin{align}
&\inference[\textbf{Eadd:}]{
    \langle e_1, m \rangle \Downarrow \langle n_1, m' \rangle &&
    \langle e_2, m' \rangle \Downarrow \langle n_2, m'' \rangle
}{
    \langle e_1 + e_2, m \rangle \Downarrow \langle n_1 + n_2, m'' \rangle
}
\\[\spacing]
&\inference[\textbf{Esign:}]{
    \langle e, m \rangle \Downarrow \langle n, m' \rangle
}{
    \langle -e, m \rangle \Downarrow \langle -n, m' \rangle
}
\\[\spacing]
&\inference[\textbf{EnotTrue:}]{
    \langle e, m \rangle \Downarrow \langle true, m' \rangle
}{
    \langle \neg e, m \rangle \Downarrow \langle false, m'  \rangle
}
\\[\spacing]
&\inference[\textbf{EnotFalse:}]{
    \langle e, m \rangle \Downarrow \langle false, m' \rangle
}{
    \langle \neg e, m \rangle \Downarrow \langle true, m'  \rangle
}
\\[\spacing]
&\inference[\textbf{EfalseAnd:}]{
    \langle e_1, m \rangle \Downarrow \langle false, m' \rangle
}{
    \langle e_1 \wedge e_2, m \rangle \Downarrow \langle false, m'  \rangle
}
\\[\spacing]
&\inference[\textbf{EtrueAndFalse:}]{
    \langle e_1, m \rangle \Downarrow \langle true, m' \rangle &
    \langle e_2, m' \rangle \Downarrow \langle false, m'' \rangle
}{
    \langle e_1 \wedge e_2, m \rangle \Downarrow \langle false, m''  \rangle
}
\\[\spacing]
&\inference[\textbf{EtrueAndTrue:}]{
    \langle e_1, m \rangle \Downarrow \langle true, m' \rangle &
    \langle e_2, m' \rangle \Downarrow \langle true, m'' \rangle
}{
    \langle e_1 \wedge e_2, m \rangle \Downarrow \langle true, m''  \rangle
}
\\[\spacing]
&\inference[\textbf{EifTrue:}]{
    \langle e, m \rangle \Downarrow \langle true, m' \rangle &
    \langle e_1, m' \rangle \Downarrow \langle v_1, m'' \rangle
}{
    \langle \textrm{if}\, e\, \textrm{then}\, e_1\, \textrm{else}\, e_2, m \rangle 
        \Downarrow \langle v_1, m''  \rangle
}
\\[\spacing]
&\inference[\textbf{EifFalse:}]{
    \langle e, m \rangle \Downarrow \langle false, m' \rangle &
    \langle e_2, m' \rangle \Downarrow \langle v_2, m'' \rangle
}{
    \langle \textrm{if}\, e\, \textrm{then}\, e_1\, \textrm{else}\, e_2, m \rangle 
        \Downarrow \langle v_2, m''  \rangle
}
\\[\spacing]
&\inference[\textbf{EltTrue:}]{
    \langle e_1, m \rangle \Downarrow \langle n_1, m' \rangle &
    \langle e_2, m' \rangle \Downarrow \langle n_2, m'' \rangle
}{
    \langle e_1 < e_2 \rangle \Downarrow \langle true, m''  \rangle
}, ~\textrm{if}~ n_1 <_{Int} n_2
\\[\spacing]
&\inference[\textbf{EltFalse:}]{
    \langle e_1, m \rangle \Downarrow \langle n_1, m' \rangle &
    \langle e_2, m' \rangle \Downarrow \langle n_2, m'' \rangle
}{
    \langle e_1 < e_2 \rangle \Downarrow \langle false, m''  \rangle
}, ~\textrm{if}~ n_1 >_{Int} n_2
\\[\spacing]
&\inference[\textbf{Evar:}]{}{
    \langle x, m \rangle \Downarrow \langle m(x) \rangle
}, ~\textrm{if}~ m(x) \neq \bot
\\[\spacing]
&\inference[\textbf{Eappl:}]{
    \langle e_1, m \rangle \Downarrow \langle \lambda x : \tau . e , m' \rangle & \hfill\\
    \langle e_2, m' \rangle \Downarrow \langle v_2, m'' \rangle &
    \langle e[x \mapsto v_2], m'' \rangle \Downarrow \langle v_3, m''' \rangle
}{
    \langle e_1 \, e_2, m \rangle \Downarrow \langle v_3, m'''  \rangle
}
\\[\spacing]
&\inference[\textbf{Ealloc:}]{
    \langle e, m \rangle \Downarrow \langle v, m' \rangle
}{
    \langle \textrm{ref}\, e, m \rangle \Downarrow \langle \textrm{loc}_i, m\{i \mapsto v\}  \rangle
},~\textrm{where}~ i = \textrm{max}(\textrm{dom}(m)) + 1
\\[\spacing]
&\inference[\textbf{Ederef:}]{
    \langle e, m \rangle \Downarrow \langle \textrm{loc}_i, m' \rangle
}{
    \langle !e, m \rangle \Downarrow \langle m'(\textrm{loc}_i), m'  \rangle
},~\textrm{if}~ m'(\textrm{loc}_i) \neq \bot
\\[\spacing]
&\inference[\textbf{Ealias:}]{\label{eq:Ealias}
    \langle e_1, m \rangle \Downarrow \langle \textrm{loc}_i, m' \rangle &
    \langle e_2, m' \rangle \Downarrow \langle v, m'' \rangle
}{
    \langle e_1 := e_2, m \rangle \Downarrow \langle (\,), m''\{i \mapsto v\}  \rangle
}, \star
\end{align}

\textgreek{Στον κανόνα για την συνωνυμία} (aliasing) \ref{eq:Ealias} 
\textgreek{η τιμή στην οποία αποτιμάται η έκφραση είναι η $(\,)$, ή αλλιώς
$unit$.}
% \textgreek{επιλέγουμε η τιμή στην οποία αποτιμάται η έκφραση να είναι η 
% τιμή στην οποία αποτιμήθηκε το δεξί μέλος. 
% Την επιλογή στηρίζουμε στον τρόπο αποτίμησης μιας τέτοιας έκφρασης στην \
% γλώσσα} C. \textgreek{Είναι μια λογική επιλογή καθώς, αν δεν γίνεται κάπως έτσι,
% η γλώσσα δεν θα μπορεί να σειροποιεί εκφράσεις} (sequencing)\textgreek{, 
% γιατί δεν έχει κάτι όπως το} ; \textgreek{της} C.
% \textgreek{Η μόνη επιλογή είναι μέσω της σύνθεσης.}

\section{\textgreek{Ασφάλεια}}
\textgreek{Το θεώρημα της ασφάλειας μιας γλώσσας προγραμματισμού είναι ισοδύναμο
με την συνέπεια του συστήματος τύπων. Η διατύπωση της ασφάλειας απαρτίζεται
από την διατύπωση του θεωρήματος προόδου και του θεωρήματος διατήρησης
για το σύστημα τύπων.
Η ``απλοϊκή'', αρχική μορφή των θεωρημάτων παρατίθεται παρακάτω.
Η διατύπωση για την υποκείμενη γλώσσα γίνεται
στην ενότητα~\ref{sec:preservation}.}

\smallskip
\noindent
\textbf{\textgreek{Θεώρημα Προόδου} (Progress):}
\textgreek{Αν $e : \tau$ τότε είτε $e$ είναι τιμή, είτε υπάρχει $e'$ τέτοιο
ώστε $e -> e'$.}

\smallskip
\noindent
\textbf{\textgreek{Θεώρημα Διατήρησης} (Preservation):}
\textgreek{Αν $e : \tau$ και $e -> e'$ τότε $e' : \tau$.}

\smallskip
\textgreek{Αυτό που επιτυγχάνουν τα 2 θεωρήματα μαζί είναι ότι αν η έκφραση $e$ έχει τύπο,
τότε η αποτίμησή της δεν μπορεί να κολλήσει.}

\subsection{\textgreek{Κανόνες τύπων}}
\textgreek{Ξεκινάμε διατυπώνοντας τους κανόνες των τύπων που 
προκύπτουν με βάση την σημασιολογία στην 
ενότητα~\ref{sec:big-step-sos} και τις διαφάνειες.
Αρχικά για τις τιμές και έπειτα για τους συμπερασματικούς κανόνες των 
εκφράσεων.
Η σημειογραφία $e : \tau$, σημαίνει ότι η έκφραση $e$ έχει τύπο $\tau$.
Με $\Gamma$ συμβολίζουμε τα περιβάλλοντα τύπων (ζεύγη $(x, \tau)$) ενώ
με $M$ συμβολίζουμε τύπους μνήμης (π.χ. $\mathbb{N} \rightharpoonup \tau$).}


\begin{align*}
    \inference[]{}{
        n : \textrm{Int}
    }
    &&
    \inference[]{}{
        true : \textrm{Bool}
    }
    &&
    \inference[]{}{
        false : \textrm{Bool}
    }
\end{align*}
\begin{align*}
    \inference[]{
        \Gamma \cup \{x : \tau_1\} |- e : \tau_2
    }{
        \Gamma |-~ \lambda x:\tau_1 ~.~ e  : \tau_1 -> \tau_2
    }&&
    \inference[]{
        M(i) : \tau
    }{
        \Gamma; M |-~ \textrm{loc}_i : \textrm{Ref}\, \tau
    }
\end{align*}

\vspace{2em}
\begin{align}
&\inference[\textbf{Tadd:}]{
    \Gamma |- e_1 : \textrm{Int}  &&
    \Gamma |- e_2 : \textrm{Int}
}{
    \Gamma |- e_1 + e_2 : \textrm{Int}
}
\\[\spacing]
&\inference[\textbf{Tsign:}]{
    \Gamma |- e : \textrm{Int}
}{
    \Gamma |- -e : \textrm{Int}
}
\\[\spacing]
&\inference[\textbf{Tnot:}]{
    \Gamma |- e : \textrm{Bool}
}{
    \Gamma |- \neg e : \textrm{Bool}
}
\\[\spacing]
&\inference[\textbf{Tand:}]{
    \Gamma |- e_1 : \textrm{Bool} &
    \Gamma |- e_2 : \textrm{Bool}
}{
    \Gamma |-  e_1 \wedge e_2 : \textrm{Bool}
}
\\[\spacing]
&\inference[\textbf{Tif:}]{
    \Gamma |- e : \textrm{Bool} &
    \Gamma |- e_1 : \tau &
    \Gamma |- e_2 : \tau
}{
    \Gamma |- \textrm{if}\, e\, \textrm{then}\, e_1\, \textrm{else}\, e_2 : \tau
}
\\[\spacing]
&\inference[\textbf{Tlt:}]{
    \Gamma |- e_1 : \textrm{Int} &
    \Gamma |- e_2 : \textrm{Int}
}{
    \Gamma |- e_1 < e_2 : \textrm{Bool}
}
\\[\spacing]
&\inference[\textbf{Tvar:}]{
    (x,\tau) \in \Gamma
}{
    \Gamma |- x : \tau
}
\\[\spacing]
&\inference[\textbf{Tappl:}]{
    \Gamma |- e_1 : \tau_1 -> \tau_2 &
    \Gamma |- e_2 : \tau_1
}{
    \Gamma |- e_1 \, e_2: \tau_2
}
\\[\spacing]
&\inference[\textbf{Talloc:}]{
    \Gamma |- e : \tau
}{
    \Gamma |- \textrm{ref}\, e : \textrm{Ref}\, \tau
}
\\[\spacing]
&\inference[\textbf{Tderef:}]{
    \Gamma |- e : \textrm{Ref}\, \tau
}{
    \Gamma |-~ ! e : \tau
}
\\[\spacing]
&\inference[\textbf{Talias:}]{\label{eq:Talias}
    \Gamma |- e_1 : \textrm{Ref}\, \tau &
    \Gamma |- e_2 : \tau
}{
    \Gamma |- e_1 := e_2 : (\,)
}
\end{align}

\subsection{\textgreek{Διατύπωση των θεωρημάτων}}
\label{sec:progress}
\label{sec:preservation}
\textgreek{Το θεώρημα κανονικοποίησης (ή τερματισμού) δεν
ισχύει διότι το σύστημα τύπων μπορεί να εμπεριέχει αναδρομή.
Επομένως, δεν μπορούμε να δηλώσουμε ότι για μια έκφραση 
$e : \tau$ θα υπάρχει πάντα τιμή $v$ έτσι ώστε 
$\langle e, m \rangle \Downarrow \langle v, m' \rangle$.}

\noindent
\textgreek{\textbf{Διατύπωση της Προόδου.}}
\textgreek{Για την πρόοδο μπορούμε να πούμε, με βάση τους κανόνες
στις προηγούμενες ενότητες, ότι;εαν 
\begin{itemize}
    \setlength\itemsep{-0.2em}
    \item $\varnothing ; M |- e : \tau$ και\
    \item $\forall v, \neg ( \langle e, m \rangle \Downarrow \langle v, m' \rangle)$
\end{itemize}
τότε $\langle e, m \rangle \Downarrow_{\infty}$. Από αυτό ακολουθεί ότι;
εαν
$\varnothing; M |- e : \tau$, τότε είτε
$\langle e, m \rangle \Downarrow_{\infty}$
είτε υπάρχει τιμή $v$ τέτοια ώστε
$\langle e, m \rangle \Downarrow \langle v, m' \rangle$~\cite{leroy2009coinductive}.
}


\textgreek{Σημειώνουμε $\Gamma |- m : M$ όταν}
$\textrm{dom}(m) = \textrm{dom}(M)$
\textgreek{και για κάθε}
$i \in \textrm{dom}(m)$
\textgreek{υπάρχει}
$\Gamma ; M |- m(i) : M(i)$.

\noindent
\textgreek{\textbf{Διατύπωση της Διατήρησης.}}
\textgreek{Για την διατήρηση μπορούμε να πούμε ότι;
εαν 
\begin{itemize}
    \setlength\itemsep{-0.2em}
    \item $\Gamma; M |- e : \tau$,
    \item $\Gamma |- m : M$, και
    \item $\langle e, m \rangle \Downarrow \langle v, m' \rangle$,
\end{itemize}
τότε υπάρχει κάποιο $M'$ τέτοιο ώστε,
\begin{itemize}
    \setlength\itemsep{-0.2em}
    \item $M \sqsubseteq M'$,
    \item $\Gamma |- m' : M'$, και
    \item $\Gamma ; M' |- v : \tau$.
\end{itemize}}


\section{\textgreek{Σύγκριση}}
\textgreek{Σε αυτήν την ενότητα συγκρίνουμε την σημασιολογία
μικρών βημάτων, που αναπτύχθηκε στις διαλέξεις και βρίσκεται
στις διαφάνειες, με αυτήν των μεγάλων βημάτων που 
χρησιμοποιήθηκε παραπάνω.

Εύκολα μπορούμε να διακρίνουμε τις διαφορές μεταξύ των 
2 προσεγγίσεων. Η σημασιολογία μεγάλων βημάτων έχει το 
πλεονέκτημα ότι συχνά καταλήγει σε απλούστερους και πιο 
ευκολονόητους, με χαμηλότερες απαιτήσεις στην σχολαστικότητα.
Αυτό μπορεί σε κάποιες περιπτώσεις να οδηγήσει σε 
απλούστερες αποδείξεις κάποιων θεωρημάτων
και συχνά αντιστοιχεί άμεσα με αποδοτική υλοποίηση
διερμηνέα για μια γλώσσα~\cite{leroy2009coinductive},
γι' αυτό λέγεται και ````φυσική'''' σημασιολογία. 
Από την άλλη, στην ενότητα~\ref{sec:preservation} έγινε
αντιληπτό ότι με την σημασιολογία μεγάλων βημάτων δεν
μπορούμε να εντοπίσουμε αποκλίνουσες αποτιμήσεις
γιατί δεν αντιστοιχούν σε κανόνες, οδηγώντας
έτσι σε μια πιο ````αμήχανη'''' διατύπωση για το θεώρημα
της προόδου.

Η σημασιολογία μικρών βημάτων, εν αντιθέσει,
ναι μεν απαιτεί να οριστούν πιο διεξοδικά οι βηματικοί κανόνες,
δίνει, δε, μεγαλύτερο έλεγχο στις λεπτομέρειες των αποτιμήσεων
(π.χ. σειρά αποτίμησης) και επιτρέπει έτσι την διατύπωση
ακριβέστερα ορισμένων θεωρημάτων για την συμπεριφορά
της γλώσσας. Έτσι, η αέναη εκτέλεση και ο μη-κανονικός
τερματισμός (κόλλημα) δύναται αναγνωρίζονται ως κάτι διαφορετικό
γιατί με κατάλληλα ορισμένους κανόνες οδηγούν σε διαφορετικά
δέντρα~\cite{nielson2007semantics}. Αυτό την κάνει πιο βολική για χρήση στην απόδειξη
του θεωρήματος της ασφάλειας.

Σε γενικές γραμμές, η επιλογή του τρόπου έκφρασης 
για την σημασιολογία μιας γλώσσας βολεύει να σχετίζεται
με τα πράγματα τα οποία θα θέλαμε να αποδείξουμε για αυτήν.
Ενώ η σημασιολογία μεγάλων βημάτων είναι συχνή στην επαλήθευση
μεταγλωττιστών, αυτή των μικρών βημάτων επιτρέπει συμβατικές
προσεγγίσεις στο θέμα της ασφάλειας μέσω λημμάτων προόδου 
και διατήρησης~\cite{owens2016functional}.}

\bigskip
\newpage
\nocite{*}
\bibliography{opsem}
\bibliographystyle{plain}
\end{document}